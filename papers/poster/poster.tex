\documentclass[dvipsnames]{beamer}
\usepackage[orientation=landscape,scale=1.6]{beamerposter}
\mode<presentation>{\usetheme{ZH}}
\usepackage{fontspec}
\usepackage{nicefrac}       % compact symbols for 1/2, etc.
\usepackage{microtype}      % microtypography
\usepackage{dsfont}
\usepackage{amsmath,amsfonts,amssymb,amsthm}
\usepackage[english]{babel} % required for rendering German special characters
\usepackage{siunitx} %pretty measurement unit rendering
\usepackage{ragged2e}
\usepackage[justification=justified]{caption}
\usepackage{array,booktabs,tabularx}
\usepackage{float}
\usepackage[plain]{algorithm}
\usepackage[noend]{algpseudocode}
\usepackage{tikz}
\usepackage{pgfplots}
\usepackage{subcaption}
\usepackage{qrcode}
\pgfplotsset{compat=1.17}
\defaultfontfeatures{Mapping=tex-text,Scale=MatchLowercase}
%\setmonofont{Iosevka SS08}

%\makeatletter
%\algrenewcommand\ALG@beginalgorithmic{\setmainfont{Iosevka SS08}}
%\makeatother


\newcommand{\sgn}{\operatorname{sgn}}
\newcommand{\conv}{\operatorname{conv}}
\newcommand{\vect}{\operatorname{vect}}
\DeclareMathOperator*{\argmin}{arg\,min}
\DeclareMathOperator*{\argmax}{arg\,max}
\newcommand*\diff{\mathop{}\!\mathrm{d}}
\newcommand{\Var}{\mathrm{Var}}
\newcommand*\ri{\mathop{}\!\mathrm{ri}}
\newcommand*\aff{\mathop{}\!\mathrm{aff}}
\newcommand*\dom{\mathop{}\!\mathrm{dom}}
\newcommand*\epi{\mathop{}\!\mathrm{epi}}
\newcommand*\diag{\mathop{}\!\mathrm{diag}}
\newcommand*\cov{\mathop{}\!\mathrm{cov}}
\newcommand*\var{\mathop{}\!\mathrm{var}}
\newcommand*\corr{\mathop{}\!\mathrm{corr}}

\newcolumntype{Z}{>{\centering\arraybackslash}X} % centered tabularx columns
\sisetup{per=frac,fraction=sfrac}

\title{Nearest Neighbour Based Estimates of Gradients: \\ Sharp Nonasymptotic Bounds and Applications}
\author{Guillaume Ausset$^{1,2}$, Stéphan Cl\'emen\c{c}on$^{1}$, Fran\c{c}ois Portier$^{1}$}
\institute{$^{1}$ T\'el\'ecom ParisTech, LTCI, Universit\'e Paris Saclay $^{2}$ BNP Paribas}

% edit this depending on how tall your header is. We should make this scaling automatic :-/
\newlength{\columnheight}
\setlength{\columnheight}{103cm}

\begin{document}
\begin{frame}
\begin{columns}
    \begin{column}{.5\textwidth}
		\begin{beamercolorbox}[center]{postercolumn}
			\begin{minipage}{.98\textwidth}  % tweaks the width, makes a new \textwidth
				\parbox[t][\columnheight]{\textwidth}{ % must be some better way to set the the height, width and textwidth simultaneously
					\begin{myblock}{Local Linear Estimation}
                        We want to estimate the gradient of a function $f$, whose gradient is supposed sparse. We suppose that we are able to gather evaluations $Y = f(X)$ of the function of interest.\newline

                        \textbf{Local Linear Estimation of the Gradient}\newline
                        If we locally approximate the function by its Taylor expansion, we can estimate the gradient by solving
                        \begin{equation*}
                            (\tilde m_{k}   (x) , \tilde \beta_k (x)) \in
                            \argmin_{(m, \beta)\in \mathbb{R}^{D+ 1}} \sum_{i : X_i \in \texttt{KNN}(x)} (Y_i - m - \beta^\intercal (X_i - x))^2
                            + \lambda \lVert \beta \rVert_1 \label{lll}
                        \end{equation*}
                        \newline
                        \begin{theorem}
                            \newline
                           Let $n\geq 1$ and $k\geq 1$ such that $\overline{\tau} _ k\leq \tau_0$.  Let $\delta\in (0,1)$ and set  $\lambda =  \overline{\tau} _ k  ( \sqrt{ 2   \sigma^2   \log(16D/\delta)/k } + L_2 \overline{\tau} _ k^2 )$. Then, we have with probability larger than $1-\delta$,
                            \begin{equation*}
                            \lVert  \tilde{\beta}_k  (x) - \beta(x)  \rVert _ 2\leq (24)^2  \sqrt{\#\mathcal S_x }    \left(  \overline \tau_k ^{-1} \sqrt{\frac{ 2   \sigma^2   \log(16D/\delta)}{k} } + L_2 \overline \tau_k  \right),
                            \end{equation*}
                            as soon as $C_1  \#\mathcal S_x \log(  D n / \delta)   \leq k  \leq  C_2  n $,   $  \overline\tau_k   ^{2}     \leq  (   b_f^2 /( C_3 \#\mathcal S_x L ^2 )  \wedge \tau_0 ^2 )$, where $C_1$, $C_2$ and $C_3$ are universal constants.
                            \end{theorem}
                            \newline\newline
                            Our bounds make use of the \emph{sparsity} of $\nabla f$ and only require a neighbourhood consisting of the $K$ nearest points.\newline\newline
                            The use of a KNN neighbourhood gives us:
                            \begin{itemize}
                                \item Robustness to the data
                                \item Ease of calibration
                                \item Possibility to reuse past computations
                            \end{itemize}
                    \end{myblock}
                    \begin{myblock}{Variable Selection}
                        \textbf{Idea:} unimportant variables will have gradients equal to zero on that dimension.
                        \begin{algorithm}[H]
                            \caption{Node Splitting for Gradient Guided Trees}\label{alg:LocalLinearTree}
                            \begin{algorithmic}[1] %
                                \Require $(X, Y)$: training set, $\texttt{Node}$: indexes of points in the node
                                \State{$\nabla m (X_i) \gets \texttt{estimated gradient at } X_i, \, \forall i \in \texttt{Node}$ \texttt{ using }~\eqref{lll}}
                                \State{$\omega \gets \sum_{i \in \texttt{Node}} \lvert \nabla m(X_i) \rvert$}
                                \State{$K \gets$ \texttt{sample} $\sqrt{D}$ \texttt{dimensions in} $\{1, \ldots, d\} \texttt{ with probability weights} \propto \omega$}
                                \State{$k, c \gets \texttt{best threshold } c \texttt{ and dimension } k$}
                                \State{$\textbf{return } k, c$}
                            \end{algorithmic}
                        \end{algorithm}
                    \end{myblock}\vfill
            }\end{minipage}\end{beamercolorbox}
    \end{column}
    	% Colonne de Droite
	\begin{column}{.5\textwidth}
		\begin{beamercolorbox}[center]{postercolumn}
			\begin{minipage}{.98\textwidth} % tweaks the width, makes a new \textwidth
				\parbox[t][\columnheight]{\textwidth}{ % must be some better way to set the the height, width and textwidth simultaneously
                    \begin{myblock}{Variable Selection}
                        \begin{table}[H]
                            \centering
                            \begin{tabular}{lrrrr}
                                & \multicolumn{2}{c}{Description} & \multicolumn{2}{c}{Loss} \\
                                \cmidrule(l){2-3} \cmidrule(l){4-5}
                                Dataset & $n$ & $D$ & RF & GGF \\
                                \midrule
                                Wisconsin & $569$ & $30$ & $0.0352$ & $\mathbf{0.0345}$ \\
                                Heart & $303$ & $13$ & $0.128$ & $\mathbf{0.124}$ \\
                                Diamonds & $53940$ & $23$ & $680033$ & $\mathbf{664265}$ \\
                                Gasoline & $60$ & $401$ & $0.678$ & $\mathbf{0.512}$ \\
                                SDSS & $10000$ & $8$ & $0.872\cdot 10^{-3}$  & $\mathbf{0.776}\cdot 10^{-3}$ \\
                                \bottomrule
                            \end{tabular}
                            \caption{Performance of the two random forest algorithms on a $50$-folds cross validation.}\label{table:results}
                        \end{table}
                    \end{myblock}
					\begin{myblock}{Gradient Free Optimization}
                        \begin{algorithm}[H]
                            \caption{Estimated Gradient Descent}\label{alg:lolamin}
                            \begin{algorithmic}[1] %
                                \Require $x_0$: initial guess, $f$: function $\mathbb{R}^D \to \mathbb{R}$, $M$: budget
                                \State{$X \gets X_1, \ldots, X_M \texttt{ with } X_i \sim \mathcal{N}(x_0, \varepsilon \times I_D)$}
                                \State{$Y \gets f(X) := f(X_1), \ldots, f(X_M)$}
                                \While{\texttt{not StoppingCondition}}
                                    \State{$m, \Delta \gets$ \texttt{estimated gradient at} $x$ \texttt{w.r.t} $X, Y$ \texttt{using}~\eqref{lll}}
                                    \State{$X \gets X, X_1, \ldots, X_M \texttt{ with } X_i \sim \mathcal{N}(\texttt{GradientStep}(x, \Delta), \varepsilon \times I_D)$}
                                    \State{$Y \gets f(X)$}
                                    \State{$x \gets \argmin_{X_i} \{ f(X_i) \}$}
                                \EndWhile
                                \State{$\textbf{return } x$}
                            \end{algorithmic}
                        \end{algorithm}
                        \begin{figure}[htb]
                            \centering
                            \begin{subfigure}[t]{.5\textwidth}
                                \begin{tikzpicture}
\begin{axis}[xlabel={\# function evaluations},width=24cm,height=15cm, ylabel={$\lvert f(x^\star) - f(x) \rvert$}, xtick={0,2500,5000}, legend style={draw=none}]
\legend{{Ours},{True $\nabla$},{Nelder-Mead}}
    \addplot+[no marks]
        table[x expr=\thisrowno{0}*50, row sep={\\}]
        {
            x  y  \\
            0.0  39699.0  \\
            1.0  38661.01222013547  \\
            2.0  38257.07661628965  \\
            3.0  37307.40555030072  \\
            4.0  36602.470842273106  \\
            5.0  35187.67089026399  \\
            6.0  32339.80554213424  \\
            7.0  31625.676311979743  \\
            8.0  27096.45117682102  \\
            9.0  22356.300295364614  \\
            10.0  21056.046623936476  \\
            11.0  20076.539398271583  \\
            12.0  17850.2852131037  \\
            13.0  17066.183387951365  \\
            14.0  17165.875858373205  \\
            15.0  16974.783045671284  \\
            16.0  16644.828990279144  \\
            17.0  17180.256171648292  \\
            18.0  16525.489917713432  \\
            19.0  15854.614862582781  \\
            20.0  10939.669669772113  \\
            21.0  7639.746669158685  \\
            22.0  6707.588190227669  \\
            23.0  6155.088924048022  \\
            24.0  6382.423708635092  \\
            25.0  5356.973320599942  \\
            26.0  5194.4206390614445  \\
            27.0  4242.043502428031  \\
            28.0  3909.536331868685  \\
            29.0  3740.7256196643966  \\
            30.0  3013.393872006049  \\
            31.0  2654.119323199185  \\
            32.0  2459.8880243732046  \\
            33.0  2065.985878970206  \\
            34.0  2154.1258265404485  \\
            35.0  2192.4021877373616  \\
            36.0  2203.628613898691  \\
            37.0  2122.413339031094  \\
            38.0  2309.38291617143  \\
            39.0  2253.4912448406862  \\
            40.0  2226.398050930897  \\
            41.0  2056.991592699621  \\
            42.0  2103.1192672383145  \\
            43.0  2050.5688941048893  \\
            44.0  1780.9849366516366  \\
            45.0  1505.9943010545148  \\
            46.0  1572.7017037559315  \\
            47.0  1658.569660939209  \\
            48.0  1758.740702274854  \\
            49.0  1491.2098560520756  \\
            50.0  1524.3858614778053  \\
            51.0  1535.5870766859675  \\
            52.0  1431.90564578395  \\
            53.0  1472.6553228053672  \\
            54.0  1493.6111687436096  \\
            55.0  1449.9719738499484  \\
            56.0  1470.6942397904957  \\
            57.0  1568.550997603763  \\
            58.0  1574.3229264539311  \\
            59.0  1602.4832036660457  \\
            60.0  1567.5408877480359  \\
            61.0  1653.5290649494857  \\
            62.0  1650.9313659397853  \\
            63.0  1693.5480324658538  \\
            64.0  1676.1196539855512  \\
            65.0  1564.6304202404772  \\
            66.0  1616.9532580748146  \\
            67.0  1651.7083854074444  \\
            68.0  1678.0067702696274  \\
            69.0  1650.4952403645864  \\
            70.0  1599.6737806855306  \\
            71.0  1563.111361569664  \\
            72.0  1602.452680047029  \\
            73.0  1598.0720288866387  \\
            74.0  1599.4361730825474  \\
            75.0  1607.5861611398384  \\
            76.0  1561.561154541469  \\
            77.0  1628.2205949300949  \\
            78.0  1662.4810842207805  \\
            79.0  1631.8231558327486  \\
            80.0  1654.2652710232778  \\
            81.0  1571.6474432727462  \\
            82.0  1562.4142535001747  \\
            83.0  1647.6329477249467  \\
            84.0  1569.7630453063884  \\
            85.0  1586.7571033939148  \\
            86.0  1533.0532205245609  \\
            87.0  1480.4316203841272  \\
            88.0  1377.8650908815393  \\
            89.0  1293.9893767798308  \\
            90.0  1248.4522276279633  \\
            91.0  1207.9070289455703  \\
            92.0  1200.5171803247856  \\
            93.0  1237.2485502685831  \\
            94.0  1296.622428896029  \\
            95.0  1345.4256942212473  \\
            96.0  1236.562856916525  \\
            97.0  1270.5819635850073  \\
            98.0  1268.3832101487653  \\
            99.0  1166.532804745176  \\
            100.0  1219.851441261067  \\
        }
        ;
    \addplot+[no marks]
        table[x expr=\thisrowno{0}*50, row sep={\\}]
        {
            x  y  \\
            0.0  39699.0  \\
            1.0  21273.580159  \\
            2.0  8361.816535773676  \\
            3.0  2528.453552575729  \\
            4.0  467.2838146610457  \\
            5.0  41.64359192538598  \\
            6.0  182.63178866081125  \\
            7.0  431.5842617044014  \\
            8.0  623.8317479395903  \\
            9.0  720.7492707978596  \\
            10.0  730.7930660476478  \\
            11.0  676.1329444211493  \\
            12.0  580.8430716739987  \\
            13.0  467.9497161883569  \\
            14.0  358.72193086904474  \\
            15.0  271.5017376129448  \\
            16.0  219.78799158105423  \\
            17.0  210.25544525591744  \\
            18.0  241.53860734598283  \\
            19.0  304.48711957723856  \\
            20.0  384.22671793795405  \\
            21.0  463.6946297553302  \\
            22.0  527.6400576173379  \\
            23.0  565.8192022869789  \\
            24.0  574.4587969030347  \\
            25.0  555.8033772717197  \\
            26.0  516.2747493060102  \\
            27.0  464.1212827767793  \\
            28.0  407.3514024125243  \\
            29.0  352.3959747387081  \\
            30.0  303.56510674815604  \\
            31.0  263.11574604503437  \\
            32.0  231.66302028502355  \\
            33.0  208.7023361756356  \\
            34.0  193.09184220873672  \\
            35.0  183.42565955920026  \\
            36.0  178.28489605846715  \\
            37.0  176.38354712772963  \\
            38.0  176.63678031676616  \\
            39.0  178.1782288824284  \\
            40.0  180.34738729193037  \\
            41.0  182.66200487449672  \\
            42.0  184.78530743975975  \\
            43.0  186.4942635344483  \\
            44.0  187.65253860863677  \\
            45.0  188.18972317000706  \\
            46.0  188.0866542481925  \\
            47.0  187.36532165560433  \\
            48.0  186.0812489136271  \\
            49.0  184.31645948965308  \\
            50.0  182.17193825463377  \\
            51.0  179.7594248459017  \\
            52.0  177.19304090498895  \\
            53.0  174.58150738719397  \\
            54.0  172.02162787171306  \\
            55.0  169.59347018999946  \\
            56.0  167.35741556303523  \\
            57.0  165.35303220732416  \\
            58.0  163.5995819050226  \\
            59.0  162.0978724073498  \\
            60.0  160.8331134474459  \\
            61.0  159.77841233260745  \\
            62.0  158.89855241241023  \\
            63.0  158.15373009403868  \\
            64.0  157.50297822756346  \\
            65.0  156.90706935873055  \\
            66.0  156.33076494295375  \\
            67.0  155.74434968457808  \\
            68.0  155.1244577715795  \\
            69.0  154.45425484470303  \\
            70.0  153.72308228575966  \\
            71.0  152.92569673336857  \\
            72.0  152.06124744803984  \\
            73.0  151.1321289089565  \\
            74.0  150.14282895920016  \\
            75.0  149.09886792394397  \\
            76.0  148.00589560222272  \\
            77.0  146.86898466050644  \\
            78.0  145.69213367816235  \\
            79.0  144.47797280836568  \\
            80.0  143.2276505803083  \\
            81.0  141.94087175663506  \\
            82.0  140.6160527142863  \\
            83.0  139.25056150205538  \\
            84.0  137.84101336328845  \\
            85.0  136.38359794331987  \\
            86.0  134.87442060947228  \\
            87.0  133.30984646229734  \\
            88.0  131.68684106484008  \\
            89.0  130.00330618105275  \\
            90.0  128.25841153924927  \\
            91.0  126.45292455084338  \\
            92.0  124.58953880311108  \\
            93.0  122.67319883916852  \\
            94.0  120.71141314184341  \\
            95.0  118.71453939109142  \\
            96.0  116.69601626020405  \\
            97.0  114.67250496374535  \\
            98.0  112.66389279060171  \\
            99.0  110.69310206179976  \\
            100.0  108.78564433474222  \\
        }
        ;
    \addplot+[no marks]
        table[x expr=\thisrowno{0}*50, row sep={\\}]
        {
            x  y  \\
            0.0  39394.0625  \\
            1.0  39394.0625  \\
            2.0  39394.0625  \\
            3.0  39394.0625  \\
            4.0  39394.0625  \\
            5.0  39394.0625  \\
            6.0  39394.0625  \\
            7.0  39394.0625  \\
            8.0  39394.0625  \\
            9.0  39394.0625  \\
            10.0  39394.0625  \\
            11.0  39394.0625  \\
            12.0  39394.0625  \\
            13.0  39394.0625  \\
            14.0  39394.0625  \\
            15.0  39394.0625  \\
            16.0  39394.0625  \\
            17.0  39274.30789875409  \\
            18.0  38861.297505809955  \\
            19.0  38736.19398394994  \\
            20.0  38558.40206737832  \\
            21.0  38142.85151860562  \\
            22.0  38053.9821995296  \\
            23.0  37900.041725084164  \\
            24.0  37498.23671284519  \\
            25.0  37414.318012602715  \\
            26.0  37284.37348967827  \\
            27.0  37135.97542465081  \\
            28.0  36885.121878349084  \\
            29.0  36716.14068281262  \\
            30.0  36566.67958463693  \\
            31.0  36300.95570826876  \\
            32.0  36204.77416164554  \\
            33.0  36012.49285673085  \\
            34.0  35690.630481721084  \\
            35.0  35677.31366360184  \\
            36.0  35503.53892295225  \\
            37.0  35339.05752031292  \\
            38.0  35083.42021978587  \\
            39.0  34924.57304411732  \\
            40.0  34819.45090356014  \\
            41.0  34511.27630089908  \\
            42.0  34423.04277448081  \\
            43.0  34325.76740911861  \\
            44.0  34177.5952944901  \\
            45.0  33933.522708610464  \\
            46.0  33830.12241985861  \\
            47.0  33694.56254240225  \\
            48.0  33359.70426564397  \\
            49.0  33332.16734520328  \\
            50.0  33198.137994731915  \\
            51.0  33048.604272654906  \\
            52.0  32831.012467805784  \\
            53.0  32648.49374090282  \\
            54.0  32549.607212853974  \\
            55.0  32270.116232076034  \\
            56.0  32223.596597502776  \\
            57.0  32076.06493505476  \\
            58.0  31738.873773904113  \\
            59.0  31725.255026324594  \\
            60.0  31614.92836996697  \\
            61.0  31414.417433247505  \\
            62.0  31263.98627715653  \\
            63.0  31107.591173260756  \\
            64.0  30980.153591604118  \\
            65.0  30765.69058462394  \\
            66.0  30670.9267500321  \\
            67.0  30540.24764192069  \\
            68.0  30259.822436661492  \\
            69.0  30234.333006608093  \\
            70.0  30066.236929886585  \\
            71.0  29743.718471152937  \\
            72.0  29743.247840416567  \\
            73.0  29607.54133264848  \\
            74.0  29468.396012573707  \\
            75.0  29274.329501902645  \\
            76.0  29160.682339442148  \\
            77.0  29023.109378307825  \\
            78.0  28768.578983576237  \\
            79.0  28720.607805105607  \\
            80.0  28583.488661506268  \\
            81.0  28288.34204402337  \\
            82.0  28248.402747576998  \\
            83.0  28120.55418230519  \\
            84.0  27986.582276632544  \\
            85.0  27816.770600641943  \\
            86.0  27711.90463322023  \\
            87.0  27453.20245275522  \\
            88.0  27415.756676668076  \\
            89.0  27250.42703549409  \\
            90.0  27145.19213217451  \\
            91.0  26975.266421052784  \\
            92.0  26847.250135575185  \\
            93.0  26698.099733069103  \\
            94.0  26579.80618749027  \\
            95.0  26385.7828929559  \\
            96.0  26299.788553834645  \\
            97.0  26159.92931084446  \\
            98.0  25998.870464535474  \\
            99.0  25887.160260825654  \\
            100.0  25721.37117494853  \\
        }
        ;
\end{axis}
\end{tikzpicture}

                            \end{subfigure}%
                            \begin{subfigure}[t]{.5\textwidth}
                                \begin{tikzpicture}
\begin{axis}[xlabel={\# function evaluations}, ylabel={$\lvert f(x^\star) - f(x) \rvert$}, width={8cm}, height={5cm}, ymax={98}, legend style={draw=none, fill=none}]
    \legend{{Ours},{Wang (2018)},{Fan (1992)}}
    \addplot+[no marks]
        table[row sep={\\}]
        {
            x  y  \\
            0.0  95.0  \\
            50.0  86.92688506338338  \\
            100.0  74.48395541925407  \\
            150.0  63.07666999954801  \\
            200.0  57.2607038503202  \\
            250.0  50.908833958660246  \\
            300.0  47.10146987100649  \\
            350.0  40.17122103123221  \\
            400.0  33.79924172687648  \\
            450.0  30.234129561848825  \\
            500.0  24.92030144424448  \\
            550.0  20.731718201964277  \\
            600.0  17.4334891945042  \\
            650.0  14.690871360980669  \\
            700.0  11.525266188442675  \\
            750.0  8.782131858275214  \\
            800.0  7.23765563663222  \\
            850.0  6.053575750504545  \\
            900.0  4.319878463569948  \\
            950.0  3.552785285314473  \\
            1000.0  2.668191277157813  \\
            1050.0  2.087978766109882  \\
            1100.0  1.8097797293656475  \\
            1150.0  1.7057423380337624  \\
            1200.0  1.2024016767168526  \\
            1250.0  1.2134766738379277  \\
            1300.0  1.3551026692913828  \\
            1350.0  1.5635914257744015  \\
            1400.0  1.404038562425725  \\
            1450.0  1.512283826130773  \\
            1500.0  1.5180532713553536  \\
            1550.0  1.3989922703863886  \\
            1600.0  1.2556566667098634  \\
            1650.0  1.2139158531453882  \\
            1700.0  1.0405147850440668  \\
            1750.0  1.0470667775136406  \\
            1800.0  0.9313196620360611  \\
            1850.0  0.8743063094119456  \\
            1900.0  0.9541325225539842  \\
            1950.0  0.8783816132947954  \\
            2000.0  0.8727423952079403  \\
            2050.0  0.9037082675249946  \\
            2100.0  0.8869478685181305  \\
            2150.0  0.9913760280049109  \\
            2200.0  0.8990486859019684  \\
            2250.0  1.0080073584876033  \\
            2300.0  1.0209423136328748  \\
            2350.0  0.907666791260809  \\
            2400.0  0.7312598769356763  \\
            2450.0  0.8679083941437422  \\
            2500.0  0.7590607390870661  \\
            2550.0  0.7292420352119223  \\
            2600.0  0.7749656655331819  \\
            2650.0  0.7322116128422672  \\
            2700.0  0.8850065602876098  \\
            2750.0  0.9995300483307634  \\
            2800.0  0.8657456861322033  \\
            2850.0  0.9276509632416583  \\
            2900.0  1.4162343589298851  \\
            2950.0  1.4340810970916464  \\
            3000.0  1.1807846700613411  \\
            3050.0  1.0264271234707196  \\
            3100.0  1.0982306242872113  \\
            3150.0  0.9675508739319226  \\
            3200.0  1.0167309760833203  \\
            3250.0  1.1038087888726797  \\
            3300.0  0.7483631119340992  \\
            3350.0  0.9778302145304075  \\
            3400.0  1.1132164227925976  \\
            3450.0  1.1944052522295425  \\
            3500.0  0.8529276342316756  \\
            3550.0  0.710275842698898  \\
            3600.0  0.8327029803277262  \\
            3650.0  0.8012208006929805  \\
            3700.0  0.7956460455884145  \\
            3750.0  1.1253822951711576  \\
            3800.0  1.3010902280381822  \\
            3850.0  1.0619714830000746  \\
            3900.0  0.817579333700858  \\
            3950.0  0.7796747596219149  \\
            4000.0  0.9604939131472778  \\
            4050.0  1.266741850661171  \\
            4100.0  0.9918111722966796  \\
            4150.0  1.2476356065932903  \\
            4200.0  1.1823151029391459  \\
            4250.0  1.2479199517869493  \\
            4300.0  1.1990574763359527  \\
            4350.0  0.9812884284147885  \\
            4400.0  0.9160726977035565  \\
            4450.0  1.1998597224934537  \\
            4500.0  1.1310133116061323  \\
            4550.0  1.0383634927460788  \\
            4600.0  0.8360838849396007  \\
            4650.0  0.7345156363149569  \\
            4700.0  0.9623472827785953  \\
            4750.0  1.2147337554275535  \\
            4800.0  1.1451648454907517  \\
            4850.0  0.9726710357680879  \\
            4900.0  1.0485780728558904  \\
            4950.0  1.1513046881730973  \\
            5000.0  1.041458912077812  \\
        }
        ;
    \addplot+[no marks]
        table[row sep={\\}]
        {
            x  y  \\
            0.0  95.0  \\
            50.0  94.45200407797681  \\
            100.0  93.85071857150601  \\
            150.0  93.12666489036995  \\
            200.0  92.00150890852889  \\
            250.0  90.87511861279825  \\
            300.0  89.6032479584178  \\
            350.0  88.18868388361417  \\
            400.0  86.77817811957738  \\
            450.0  85.34043316240255  \\
            500.0  83.74418358671826  \\
            550.0  82.13328065766962  \\
            600.0  80.59698885319995  \\
            650.0  78.86592738091638  \\
            700.0  77.15872062303356  \\
            750.0  75.38874158057942  \\
            800.0  73.7938993706989  \\
            850.0  72.26049857738177  \\
            900.0  70.7230933098338  \\
            950.0  69.25175399004395  \\
            1000.0  67.7970051424959  \\
            1050.0  66.33808178600232  \\
            1100.0  64.79425824102859  \\
            1150.0  63.26684629540381  \\
            1200.0  61.74543449113505  \\
            1250.0  60.250239895240355  \\
            1300.0  58.77228457189068  \\
            1350.0  57.251035755688164  \\
            1400.0  55.891108539119344  \\
            1450.0  54.61176115174702  \\
            1500.0  53.23873037321646  \\
            1550.0  51.98179372384748  \\
            1600.0  50.664603862471424  \\
            1650.0  49.35449567339586  \\
            1700.0  48.18964553143103  \\
            1750.0  46.979155238295625  \\
            1800.0  45.87767649260961  \\
            1850.0  44.79968712207235  \\
            1900.0  43.773925729080666  \\
            1950.0  42.78847594447637  \\
            2000.0  41.89891759608961  \\
            2050.0  41.00935223360584  \\
            2100.0  40.13993906765961  \\
            2150.0  39.25660595085529  \\
            2200.0  38.385000613722724  \\
            2250.0  37.557129866912106  \\
            2300.0  36.69624804933529  \\
            2350.0  35.883546042961925  \\
            2400.0  35.09916651199926  \\
            2450.0  34.3668255140413  \\
            2500.0  33.64518289377117  \\
            2550.0  32.93317256307173  \\
            2600.0  32.21690975484487  \\
            2650.0  31.532342288625713  \\
            2700.0  30.86918531702486  \\
            2750.0  30.211417907907265  \\
            2800.0  29.606985398099663  \\
            2850.0  28.998656153279185  \\
            2900.0  28.416791930583717  \\
            2950.0  27.836745600939825  \\
            3000.0  27.279341191573245  \\
            3050.0  26.726608706582642  \\
            3100.0  26.176482255457646  \\
            3150.0  25.65781463384676  \\
            3200.0  25.142176518024634  \\
            3250.0  24.65287486507048  \\
            3300.0  24.183415055462017  \\
            3350.0  23.738013151905033  \\
            3400.0  23.309292766969662  \\
            3450.0  22.893573445715823  \\
            3500.0  22.477399812324666  \\
            3550.0  22.080195766144303  \\
            3600.0  21.700011275882048  \\
            3650.0  21.333362614696668  \\
            3700.0  20.967571251837224  \\
            3750.0  20.607341960671935  \\
            3800.0  20.254601951530855  \\
            3850.0  19.909838226233727  \\
            3900.0  19.563883265029407  \\
            3950.0  19.224457311013715  \\
            4000.0  18.907008677459267  \\
            4050.0  18.5761451928444  \\
            4100.0  18.266985583340407  \\
            4150.0  17.957087290172073  \\
            4200.0  17.656338651715902  \\
            4250.0  17.373695825589607  \\
            4300.0  17.0959107652407  \\
            4350.0  16.815903529445933  \\
            4400.0  16.55703367553761  \\
            4450.0  16.294126107694655  \\
            4500.0  16.045158795123022  \\
            4550.0  15.801831211045615  \\
            4600.0  15.566743396909672  \\
            4650.0  15.34474028941614  \\
            4700.0  15.132502429387044  \\
            4750.0  14.912856977611396  \\
            4800.0  14.690770172531337  \\
            4850.0  14.4721592929037  \\
            4900.0  14.260094758472475  \\
            4950.0  14.048749527643475  \\
            5000.0  13.840192402082208  \\
        }
        ;
    \addplot+[no marks]
        table[row sep={\\}]
        {
            x  y  \\
            0.0  95.0  \\
            50.0  94.7592034562115  \\
            100.0  94.27486182912651  \\
            150.0  93.65545899095565  \\
            200.0  92.85543625258572  \\
            250.0  91.81865635766707  \\
            300.0  90.66371527984252  \\
            350.0  89.40748926033363  \\
            400.0  88.18816191370296  \\
            450.0  86.87770680651201  \\
            500.0  85.60956830257884  \\
            550.0  84.3262601139828  \\
            600.0  82.86416451271796  \\
            650.0  81.43468676379878  \\
            700.0  79.9460828299247  \\
            750.0  78.38141514339573  \\
            800.0  77.13401129049551  \\
            850.0  75.65707341702503  \\
            900.0  74.20583123724833  \\
            950.0  72.73901895062052  \\
            1000.0  71.24276365281531  \\
            1050.0  69.95985923912873  \\
            1100.0  68.61241862909714  \\
            1150.0  67.359720961404  \\
            1200.0  66.02945111385542  \\
            1250.0  64.6069869994266  \\
            1300.0  63.28420097054796  \\
            1350.0  62.02463795715612  \\
            1400.0  60.78989136267625  \\
            1450.0  59.61595062184901  \\
            1500.0  58.3785352960116  \\
            1550.0  57.20683481363261  \\
            1600.0  56.02213164385147  \\
            1650.0  54.898274441882684  \\
            1700.0  53.813024996672326  \\
            1750.0  52.714668990155076  \\
            1800.0  51.617752646323446  \\
            1850.0  50.53936376029601  \\
            1900.0  49.43760676015783  \\
            1950.0  48.34674835753407  \\
            2000.0  47.46237649189291  \\
            2050.0  46.52445293362856  \\
            2100.0  45.59822391565341  \\
            2150.0  44.7039962599492  \\
            2200.0  43.89949428816642  \\
            2250.0  43.05402871779469  \\
            2300.0  42.23504407326578  \\
            2350.0  41.43124002532861  \\
            2400.0  40.70379217197582  \\
            2450.0  39.966438757944495  \\
            2500.0  39.36401503503258  \\
            2550.0  38.68129860472375  \\
            2600.0  38.070748067767795  \\
            2650.0  37.46430902385679  \\
            2700.0  36.84191567926455  \\
            2750.0  36.23940611720972  \\
            2800.0  35.641042971773835  \\
            2850.0  35.074006604340994  \\
            2900.0  34.49572625852925  \\
            2950.0  33.99997320889723  \\
            3000.0  33.51839895120941  \\
            3050.0  33.02691697966133  \\
            3100.0  32.54449175563856  \\
            3150.0  32.03889904884239  \\
            3200.0  31.553100002606346  \\
            3250.0  31.057467172384577  \\
            3300.0  30.66364923625832  \\
            3350.0  30.278585336945074  \\
            3400.0  29.87753858328534  \\
            3450.0  29.540307182512763  \\
            3500.0  29.18820755650187  \\
            3550.0  28.809592580870312  \\
            3600.0  28.426238703011943  \\
            3650.0  28.062691761589473  \\
            3700.0  27.67570873750112  \\
            3750.0  27.355730844740037  \\
            3800.0  27.007744735432762  \\
            3850.0  26.638958737566746  \\
            3900.0  26.254686235399248  \\
            3950.0  25.87140548772112  \\
            4000.0  25.52271724588555  \\
            4050.0  25.163040044460182  \\
            4100.0  24.86869220546485  \\
            4150.0  24.587239054239458  \\
            4200.0  24.264322014253246  \\
            4250.0  23.94046607879286  \\
            4300.0  23.6088081137752  \\
            4350.0  23.268563806495287  \\
            4400.0  22.938141301434634  \\
            4450.0  22.606192580361757  \\
            4500.0  22.29579618181944  \\
            4550.0  22.029087760114763  \\
            4600.0  21.7222177437809  \\
            4650.0  21.449551126790173  \\
            4700.0  21.167489257495284  \\
            4750.0  20.889392866826178  \\
            4800.0  20.665917664766056  \\
            4850.0  20.4213186653504  \\
            4900.0  20.1836410043168  \\
            4950.0  19.97209423059016  \\
            5000.0  19.72951764902478  \\
        }
        ;
\end{axis}
\end{tikzpicture}

                            \end{subfigure}%
                            \caption{Gradient Descent on the sparse noisy Rosenbrock function for $d=100$.}\label{fig:rosenbrock_vs}
                        \end{figure}
                        \vspace{-3cm}
                        \raggedleft\qrcode[hyperlink,height=6cm]{https://git.sr.ht/\~aussetg/locallinear}
                    \end{myblock}
                    \vfill
            }\end{minipage}\end{beamercolorbox}
    \end{column}
\end{columns}
\end{frame}
\end{document}